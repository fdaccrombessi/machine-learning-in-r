% Options for packages loaded elsewhere
\PassOptionsToPackage{unicode}{hyperref}
\PassOptionsToPackage{hyphens}{url}
%
\documentclass[
  ignorenonframetext,
]{beamer}
\title{Machine Learning Crash Course}
\author{Tajudeen Abdulazeez}
\date{6/2/2023}

\usepackage{pgfpages}
\setbeamertemplate{caption}[numbered]
\setbeamertemplate{caption label separator}{: }
\setbeamercolor{caption name}{fg=normal text.fg}
\beamertemplatenavigationsymbolsempty
% Prevent slide breaks in the middle of a paragraph
\widowpenalties 1 10000
\raggedbottom
\setbeamertemplate{part page}{
  \centering
  \begin{beamercolorbox}[sep=16pt,center]{part title}
    \usebeamerfont{part title}\insertpart\par
  \end{beamercolorbox}
}
\setbeamertemplate{section page}{
  \centering
  \begin{beamercolorbox}[sep=12pt,center]{part title}
    \usebeamerfont{section title}\insertsection\par
  \end{beamercolorbox}
}
\setbeamertemplate{subsection page}{
  \centering
  \begin{beamercolorbox}[sep=8pt,center]{part title}
    \usebeamerfont{subsection title}\insertsubsection\par
  \end{beamercolorbox}
}
\AtBeginPart{
  \frame{\partpage}
}
\AtBeginSection{
  \ifbibliography
  \else
    \frame{\sectionpage}
  \fi
}
\AtBeginSubsection{
  \frame{\subsectionpage}
}
\usepackage{amsmath,amssymb}
\usepackage{lmodern}
\usepackage{iftex}
\ifPDFTeX
  \usepackage[T1]{fontenc}
  \usepackage[utf8]{inputenc}
  \usepackage{textcomp} % provide euro and other symbols
\else % if luatex or xetex
  \usepackage{unicode-math}
  \defaultfontfeatures{Scale=MatchLowercase}
  \defaultfontfeatures[\rmfamily]{Ligatures=TeX,Scale=1}
\fi
\usetheme[]{AnnArbor}
\usecolortheme{dolphin}
\usefonttheme{structurebold}
% Use upquote if available, for straight quotes in verbatim environments
\IfFileExists{upquote.sty}{\usepackage{upquote}}{}
\IfFileExists{microtype.sty}{% use microtype if available
  \usepackage[]{microtype}
  \UseMicrotypeSet[protrusion]{basicmath} % disable protrusion for tt fonts
}{}
\makeatletter
\@ifundefined{KOMAClassName}{% if non-KOMA class
  \IfFileExists{parskip.sty}{%
    \usepackage{parskip}
  }{% else
    \setlength{\parindent}{0pt}
    \setlength{\parskip}{6pt plus 2pt minus 1pt}}
}{% if KOMA class
  \KOMAoptions{parskip=half}}
\makeatother
\usepackage{xcolor}
\IfFileExists{xurl.sty}{\usepackage{xurl}}{} % add URL line breaks if available
\IfFileExists{bookmark.sty}{\usepackage{bookmark}}{\usepackage{hyperref}}
\hypersetup{
  pdftitle={Machine Learning Crash Course},
  pdfauthor={Tajudeen Abdulazeez},
  hidelinks,
  pdfcreator={LaTeX via pandoc}}
\urlstyle{same} % disable monospaced font for URLs
\newif\ifbibliography
\usepackage{color}
\usepackage{fancyvrb}
\newcommand{\VerbBar}{|}
\newcommand{\VERB}{\Verb[commandchars=\\\{\}]}
\DefineVerbatimEnvironment{Highlighting}{Verbatim}{commandchars=\\\{\}}
% Add ',fontsize=\small' for more characters per line
\usepackage{framed}
\definecolor{shadecolor}{RGB}{248,248,248}
\newenvironment{Shaded}{\begin{snugshade}}{\end{snugshade}}
\newcommand{\AlertTok}[1]{\textcolor[rgb]{0.94,0.16,0.16}{#1}}
\newcommand{\AnnotationTok}[1]{\textcolor[rgb]{0.56,0.35,0.01}{\textbf{\textit{#1}}}}
\newcommand{\AttributeTok}[1]{\textcolor[rgb]{0.77,0.63,0.00}{#1}}
\newcommand{\BaseNTok}[1]{\textcolor[rgb]{0.00,0.00,0.81}{#1}}
\newcommand{\BuiltInTok}[1]{#1}
\newcommand{\CharTok}[1]{\textcolor[rgb]{0.31,0.60,0.02}{#1}}
\newcommand{\CommentTok}[1]{\textcolor[rgb]{0.56,0.35,0.01}{\textit{#1}}}
\newcommand{\CommentVarTok}[1]{\textcolor[rgb]{0.56,0.35,0.01}{\textbf{\textit{#1}}}}
\newcommand{\ConstantTok}[1]{\textcolor[rgb]{0.00,0.00,0.00}{#1}}
\newcommand{\ControlFlowTok}[1]{\textcolor[rgb]{0.13,0.29,0.53}{\textbf{#1}}}
\newcommand{\DataTypeTok}[1]{\textcolor[rgb]{0.13,0.29,0.53}{#1}}
\newcommand{\DecValTok}[1]{\textcolor[rgb]{0.00,0.00,0.81}{#1}}
\newcommand{\DocumentationTok}[1]{\textcolor[rgb]{0.56,0.35,0.01}{\textbf{\textit{#1}}}}
\newcommand{\ErrorTok}[1]{\textcolor[rgb]{0.64,0.00,0.00}{\textbf{#1}}}
\newcommand{\ExtensionTok}[1]{#1}
\newcommand{\FloatTok}[1]{\textcolor[rgb]{0.00,0.00,0.81}{#1}}
\newcommand{\FunctionTok}[1]{\textcolor[rgb]{0.00,0.00,0.00}{#1}}
\newcommand{\ImportTok}[1]{#1}
\newcommand{\InformationTok}[1]{\textcolor[rgb]{0.56,0.35,0.01}{\textbf{\textit{#1}}}}
\newcommand{\KeywordTok}[1]{\textcolor[rgb]{0.13,0.29,0.53}{\textbf{#1}}}
\newcommand{\NormalTok}[1]{#1}
\newcommand{\OperatorTok}[1]{\textcolor[rgb]{0.81,0.36,0.00}{\textbf{#1}}}
\newcommand{\OtherTok}[1]{\textcolor[rgb]{0.56,0.35,0.01}{#1}}
\newcommand{\PreprocessorTok}[1]{\textcolor[rgb]{0.56,0.35,0.01}{\textit{#1}}}
\newcommand{\RegionMarkerTok}[1]{#1}
\newcommand{\SpecialCharTok}[1]{\textcolor[rgb]{0.00,0.00,0.00}{#1}}
\newcommand{\SpecialStringTok}[1]{\textcolor[rgb]{0.31,0.60,0.02}{#1}}
\newcommand{\StringTok}[1]{\textcolor[rgb]{0.31,0.60,0.02}{#1}}
\newcommand{\VariableTok}[1]{\textcolor[rgb]{0.00,0.00,0.00}{#1}}
\newcommand{\VerbatimStringTok}[1]{\textcolor[rgb]{0.31,0.60,0.02}{#1}}
\newcommand{\WarningTok}[1]{\textcolor[rgb]{0.56,0.35,0.01}{\textbf{\textit{#1}}}}
\usepackage{graphicx}
\makeatletter
\def\maxwidth{\ifdim\Gin@nat@width>\linewidth\linewidth\else\Gin@nat@width\fi}
\def\maxheight{\ifdim\Gin@nat@height>\textheight\textheight\else\Gin@nat@height\fi}
\makeatother
% Scale images if necessary, so that they will not overflow the page
% margins by default, and it is still possible to overwrite the defaults
% using explicit options in \includegraphics[width, height, ...]{}
\setkeys{Gin}{width=\maxwidth,height=\maxheight,keepaspectratio}
% Set default figure placement to htbp
\makeatletter
\def\fps@figure{htbp}
\makeatother
\setlength{\emergencystretch}{3em} % prevent overfull lines
\providecommand{\tightlist}{%
  \setlength{\itemsep}{0pt}\setlength{\parskip}{0pt}}
\setcounter{secnumdepth}{-\maxdimen} % remove section numbering
\ifLuaTeX
  \usepackage{selnolig}  % disable illegal ligatures
\fi

\begin{document}
\frame{\titlepage}

\begin{frame}{About me}
\protect\hypertarget{about-me}{}
\begin{itemize}
\tightlist
\item
  Name: Tajudeen Abdulazeez
\item
  LInkedin profile: :
  \url{https://www.linkedin.com/in/tajudeenolarewajuabdulazeez}

  This text is red
\end{itemize}

\includegraphics{logo-2.png} - Bullet 1 - Bullet 2 - Bullet 3
\end{frame}

\begin{frame}{About Sparks of Africa}
\protect\hypertarget{about-sparks-of-africa}{}
Sparks of Africa is an independent non-profit organization incessantly
working to improve and positively influence the socio-economic wellbeing
of the Nigerian people. The organization is focused on socio economic
uplift of the nation through efforts to engender empowering of the youth
of Nigeria. Nigeria has some of the largest young population in the
world and to encourage and empower the youth to be productive in
whatever career path they choose to follow will ensue a real change.
Sparks of Africa also seeks to empower youth in the area of innovation,
governance , the advancement of the economy and promotion of human and
children's right. We are also deeply involved in advocacy for youth
development and youth participation in government and in nation
building.
\end{frame}

\begin{frame}{Follow our pages}
\protect\hypertarget{follow-our-pages}{}
Facebook: \url{https://www.facebook.com/sparksofafrica} Linkedin:
\url{https://www.linkedin.com/company/sparks-of-africa/} Discord
Channel: \url{https://discord.gg/tJtHDA9w}

Website: \url{https://sparksofafrica.org/}
\end{frame}

\begin{frame}{Setting Up Development Environment}
\protect\hypertarget{setting-up-development-environment}{}
\begin{itemize}
\item
  Installing R Studio MAC PC Windows 10
\item
  HOW TO Install Packages on to R
\item
  HOW TO Set a ``Working Directory'' in R
\item
  HOW TO ``Get Data into R'' (Part 1)
\item
  HOW TO ``Get Data into R'' (Excel)
\item
  HOW TO ``Get Data into R'' (Part 2)
\end{itemize}
\end{frame}

\begin{frame}{Course Outline}
\protect\hypertarget{course-outline}{}
Day 1:~ June 3, 2023. (8am - 11am Pacific Time / 4pm - 7pm~West Africa
Standard Time) ~ ~ ~ ~ ~ ~ ~ ~ ~ ~ ~ ~Setting up development environment
~ ~ ~ ~ ~ ~ ~ ~ ~ ~ ~ ~Introduction to R programming~ ~ ~ ~ ~ ~ ~ ~ ~ ~
~ ~ ~Understanding the fundamentals of machine learning and its
importance in today's world. ~ ~ ~ ~ ~ ~ ~ ~ ~ ~ ~ ~Data Preprocessing ~
~ ~ ~ ~ ~ ~ ~ ~ ~ ~ ~Real-world applications: Case study  Day 2~: June
10, 2023.~(8am - 11am Pacific Time / 4pm - 7pm~West Africa Standard
Time)~ ~ ~ ~ ~ ~ ~ ~ ~ ~ ~ ~~Supervised Learning Algorithms ~ ~ ~ ~ ~ ~
~ ~ ~ ~ ~ ~ ~ ~ ~ ~ ~Regression ~ ~ ~ ~ ~ ~ ~ ~ ~ ~ ~ ~ ~ ~ ~ ~
~Classification ~ ~ ~ ~ ~ ~ ~ ~ ~ ~ ~Unsupervised Learning Algorithms ~
~ ~ ~ ~ ~ ~ ~ ~ ~ ~ ~ ~ ~ ~ ~ ~Clustering ~ ~ ~ ~ ~ ~ ~ ~ ~ ~ ~Model
Evaluation ~ ~ ~ ~ ~ ~ ~ ~ ~ ~ ~Real-world applications:~Case study 
Day 3: June 17, 2023.~(8am - 11am Pacific Time / 4pm - 7pm~West Africa
Standard Time) ~ ~ ~ ~ ~ ~ ~ ~ ~ ~ ~Advanced Topics in Machine Learning
in R ~ ~ ~ ~ ~ ~ ~ ~ ~ ~ ~ ~ ~ ~ ~ ~ ~Association Rule Mining Support
Vector Machines (SVMs): Understanding SVMs and building models in R
Network Graph ~ ~ ~ ~ ~ ~ ~ ~ ~ ~~Applications of Machine Learning in R
~ ~ ~ ~ ~ ~ ~ ~ ~ ~ ~ ~ ~ ~ ~ ~~Time Series Analysis: Understanding time
series analysis and building models in R. ~ ~ ~ ~ ~ ~ ~ ~ ~ ~ ~ ~ ~ ~ ~
~ Text Analytics: Understanding text analytics and building models in R
~ ~ ~ ~ ~ ~ ~ ~ ~ ~ ~ ~ ~ ~ ~ ~ ~Recommender Systems: Understanding
recommender systems and building models in R.  Day 4: June 24,
2023.~(8am - 11am Pacific Time / 4pm - 7pm~West Africa Standard Time) ~
~ ~ ~ ~ ~ ~ ~ ~ ~ ~Building an end-to-end machine learning pipeline:
From data preprocessing to model evaluation and selection. ~ ~ ~ ~ ~ ~ ~
~ ~ ~~Best practices for machine learning in R. ~ ~ ~ ~ ~ ~ ~ ~ ~
~~Future directions and advanced topics in machine learning in R.
\end{frame}

\begin{frame}{R Markdown}
\protect\hypertarget{r-markdown}{}
This is an R Markdown presentation. Markdown is a simple formatting
syntax for authoring HTML, PDF, and MS Word documents. For more details
on using R Markdown see \url{http://rmarkdown.rstudio.com}.

When you click the \textbf{Knit} button a document will be generated
that includes both content as well as the output of any embedded R code
chunks within the document.
\end{frame}

\begin{frame}{Slide with Bullets}
\protect\hypertarget{slide-with-bullets}{}
\begin{itemize}
\tightlist
\item
  Bullet 1
\item
  Bullet 2
\item
  Bullet 3
\end{itemize}
\end{frame}

\begin{frame}[fragile]{Slide with R Output}
\protect\hypertarget{slide-with-r-output}{}
\begin{Shaded}
\begin{Highlighting}[]
\FunctionTok{summary}\NormalTok{(cars)}
\end{Highlighting}
\end{Shaded}

\begin{verbatim}
##      speed           dist       
##  Min.   : 4.0   Min.   :  2.00  
##  1st Qu.:12.0   1st Qu.: 26.00  
##  Median :15.0   Median : 36.00  
##  Mean   :15.4   Mean   : 42.98  
##  3rd Qu.:19.0   3rd Qu.: 56.00  
##  Max.   :25.0   Max.   :120.00
\end{verbatim}
\end{frame}

\begin{frame}{Slide with Plot}
\protect\hypertarget{slide-with-plot}{}
\includegraphics{intro_files/figure-beamer/pressure-1.pdf}
\end{frame}

\end{document}
